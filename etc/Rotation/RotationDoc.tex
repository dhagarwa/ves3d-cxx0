\documentclass[11pt]{article}

\usepackage{algorithmic,algorithm,amsmath,amsfonts}
\newlength\commLen
\newlength\textLen
\newlength\resLen
\newcommand{\defeq}{:=}
\renewcommand{\algorithmiccomment}[1]{
        \settowidth\textLen{\footnotesize\tt// #1}
        \setlength\resLen{\the\commLen}
        \addtolength\resLen{-\the\textLen} 
        \hfill{\footnotesize\tt// #1}\hspace{\the\resLen}}

\begin{document}

``Rotation'' is one of the main kernels in our computation. It
involves a series of matrix-matrix multiplications. We're interested
in both single and double precision computations.

The input matrices are $A^{i,0} (i=0, \dots, p)$, and $B$. The output
matrix is $C$.  Here, $p$ is a ``mesh-size'' per red blood cell (RBC),
an $N$ is the number of RBCs.  The matrices $A^{i,0}$ are precomputed
and known, while the matrix $B$ stores the position of points on RBCs
and changes at every time step.

Let $M \defeq 2p(p+1)$. Then each $A^{i,0} \in \mathbb{R}^{M\times
  M}$, and $B, C \in \mathbb{R}^{M\times N}$.


The pseudocode below summarizes the algorithm:

\setlength\commLen{250pt} 
\begin{algorithm}[!hbt]
  %\caption{.}
  \begin{algorithmic}
    \FOR{$i=0$ to $p$}
    \FOR{$j=0$ to $2p-1$}
    
    \STATE $A^{i,j} \leftarrow$ permute $A^{i,0}$ 
    \STATE $C\leftarrow A^{i,j} B$

    \STATE Process $C$
    
    \ENDFOR
    \ENDFOR
    
  \end{algorithmic}
\end{algorithm}

In this double loop, there is a permutation step, in which $A^{i,0}$
is permuted to $A^{i,j}$ and then it is applied to $B$.  This
permutation step involves exchanging columns of $A^i$. For any fixed
$j\;(j=0,\dots,2p-1)$ it is defined as

\begin{equation}
A^{i,j}(n,m) = A^{i,0}\left(\, n, \left\lfloor \frac{m}{2p} \right\rfloor +
  (m\mod 2p - j)\, \right), \forall n,m.
\end{equation}
Here we're using Matlab's index notation.  Also, $m\mod 2p - j$ must
be positive between $0$ and $2p-1$. When it is negative it is shifted
by $2p$.

In our current implementation, we loop sequentially over $i$ and $j$,
we form $A^{i,j}$ and then we multiply with $B$ calling GEMM.

Possible optimizations include exploring parallelism in the $j$-index
and using equation (1) in a matrix-free matrix-matrix multiplication
instead of explicitly forming $A^{i,j}$. In addition, there is
parallelism in $i$ but the memory costs are extensive. Typical values
of $N$ and $p$ are 30--3000 and  6--20 respectively.

\end{document}

